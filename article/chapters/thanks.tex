\chapter*{謝辞}
\addcontentsline{toc}{chapter}{謝辞}
本研究を行うにあたり,ご多忙の中,終始適切かつ丁寧なご指導をして下さった大須賀昭彦教授,田原康之准教授,清雄一准教授に深く感謝致します.貴重な勉学の機会を与えてくださったことに深く御礼申し上げます.
また,研究の機会と議論・研鑽の場を提供して頂き,ご指導頂いた国立情報学研究所/東京大学の本位田真一教授をはじめ活発な議論と貴重なご意見を頂いた研究グループの皆様,大須賀・田原研究室の皆様に感謝の意を表します.

%\chapter*{研究業績}
%\addcontentsline{toc}{chapter}{研究業績}

%\section*{国際会議}
%\begin{achievement}
%\item \underline{\textbf{Minato Sato}}, Ryohei Orihara, Sei Yuichi, Yasuyuki Tahara and Akihiko Ohsuga: Japanese Text Classification by Character-Level Deep ConvNets and Transfer Learning, The 9th International Coneference on Agents and Artificial Intelligence (ICAART2017), Feb 2017. (accepted as a Full Paper)
%\end{achievement}

%\section*{査読付き国内シンポジウム・ワークショップ}
%\begin{achievement}
%\item \underline{\textbf{佐藤挙斗}},折原良平,清雄一,田原康之,大須賀昭彦: 文字レベル深層学習による日本語テキストの分類と転移学習,合同エージェントワークショップ&シンポジウム2016 (JAWS2016),pp.199-206,2016年9月. (ショート発表) \textcolor{red}{{\bf 優秀発表賞}}
%\end{achievement}

%\section*{研究会}
%\begin{achievement}
%\item \underline{\textbf{佐藤挙斗}},折原良平,清雄一,田原康之,大須賀昭彦: 文字レベル深層学習の日本語データセットへの応用,第184回 情報処理学会 知能システム研究会 (SIG-ICS) ,2016年8月.
%\item \underline{\textbf{佐藤挙斗}},折原良平,清雄一,田原康之,大須賀昭彦: 文字レベル深層学習によるテキスト分類と転移学習,人工知能学会 第102回人工知能基本問題研究会(SIG-FPAI),2016年12月. 
%\end{achievement}