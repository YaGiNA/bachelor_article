% 背景[フェイクニュースの自動判別の必要性+ジョークニュースを区別する必要性]+想定環境の説明(導入との被りに注意)
\chapter{序論}
%
\section{背景}
% 背景

%想定環境

\section{先行研究}
%フェイクニュースのテキストによる分類
フェイクニュースに限らず,風評やwebページの信憑性を評価するモデルの構築の研究は数多く行われている.
例えば,福島らの研究\cite{fuk}では,webページの体裁から信頼性を評価するモデルが提案されている.
また,機械学習による分類が非常に盛んに行われている.
なかでもGranikらの研究\cite{gra}やGildaの研究\cite{gil},そして松尾の研究\cite{mat}により,
ナイーブベイズ分類器やSVM,決定木といった教師あり学習によるフェイクニュースや流言を分類するタスクにおいて,
優秀な分類成果を挙げることが報告されている.
ほかにもWuらの研究\cite{wu}によると,SNS上で拡散された情報に対して,
``誰が・どのような経緯で拡散したか''という情報から信憑性を判断するモデルも提案されている. 

%マルチメディアでフェイクニュース分類

\section{課題}
上記のEANNモデルのような画像・テキスト双方を扱うモデルでは,実際に真実・フェイクとのカテゴリ分類において画像単独・テキスト単独の分類に比べて優秀な成績を収めている\cite{eann}.\@
しかしながら,あくまで``真実なのかそうでないのか''という2カテゴリで分類しているため,
``他者を欺くための情報なのか,皮肉・風刺を込めた情報なのか''という観点での分析がなされていない.

%本研究
本研究では,画像つきで発信された情報に対して,正しい情報か・フェイクニュースか・ジョークニュースかを判断するモデルを構築する.
このモデルを使い,従来から画像・テキスト複合のデータセットに対して3カテゴリでも優秀な分類が行えることを示すことを目指す.
それにより,SNSユーザの情報収集を支援するエージェントの開発につなげることが可能となる.

%実験
上記の提案する情報分類システムを検証するために,
事前に用意されたデータセットを用いて10 分割交差検定によって分析を行う.
また上記システムの分類性能を評価するために,
画像・テキスト単独で分類を行った結果と比較することで,
提案システムが目標に適していることを示す.

% \bibliography{../ref/ref}
