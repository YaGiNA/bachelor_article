% 背景[フェイクニュースの自動判別の必要性+ジョークニュースを区別する必要性]+想定環境の説明(導入との被りに注意)
\chapter{序論}
%
\section{背景}
% フェイクニュース
昨今のSNSの普及により,誰もが情報を発信・収集できるようになった.
特に最近ではテキストのみならず,画像や動画と併せて情報の発信が可能である.
一般論として,テキスト単体と比べて画像や動画と併せて発信されたマルチメディア情報の方が多くの注目を得やすい.
逆にこれを利用して,故意に情報を捏造して発信することによって人々を誤った方向へ扇動するフェイクニュースも存在する.
フェイクニュースが広まると、大規模なマイナスの影響が出る可能性があり、
場合によっては重要な公共の出来事に影響を及ぼしたり、操作したりすることさえある.
例えば2016年の米国大統領選では,2名の候補者を支持させるためのフェイクニュースが多く拡散され,
とりわけFacebook上では3700万回以上共有された\cite{allcott2017social}.

% ジョークニュース
虚偽の情報ながら,扇動ではなく皮肉や風刺を込めたジョークニュースも存在する.
有名な発信メディアとしては,英語ではthe Onion,日本語では虚構新聞が該当する.
あくまで扇動ではなく笑いを提供するためのものであり,
多くの場合それは批判の的にはなりにくい.
しかしながら,ジョークニュースはフェイクニュースと同じく限りなく真実を模した形式をとるため,
同じくSNS上で拡散されやすい傾向にある.

% 想定環境
当研究では,扇動のために故意に情報を捏造して発信された情報をフェイクニュース,
事実を発信した情報を正しいニュース,
そして風刺や皮肉を込めて発信された情報をジョークニュースとして定義する.

\section{先行研究}
% フェイクニュースのテキストによる分類
フェイクニュースに限らず,風評やwebページの信憑性を評価するモデルの構築の研究は数多く行われている.
例えば,福島らの研究\cite{福島隆寛2007web}では,webページの体裁から信頼性を評価するモデルが提案されている.
また,機械学習による分類が非常に盛んに行われている.
なかでもGranikらの研究\cite{Granik8100379}やGildaの研究\cite{Gilda8305411},そして松尾の研究\cite{松尾省吾2018master}により,
単語埋め込みとナイーブベイズ分類器やSVM,決定木といった教師あり学習を組み合わせることによって,
フェイクニュースや流言を分類するタスクで優秀な分類成果を挙げることが報告されている.
ほかにもWuらの研究\cite{wu2018tracing}によると,SNS上で拡散された情報に対して,
``誰が・どのような経緯で拡散したか''という情報から信憑性を判断するモデルも提案されている. 
Rubinらの研究\cite{rubin2016fake}によれば,正しいニュース・ジョークニュースの分類にもこのアプローチが有効であることが示されている.
正しいニュース・フェイクニュース・ジョークニュースの3カテゴリ分類においても研究が行われている.
特にHorneとSibelの研究\cite{horne2017just}によると,フェイクニュースは正しいニュースよりジョークニュースに近い性質をもち,
真実に近い形式をとるほど高い説得力をもつことが示されている.

% マルチメディアでフェイクニュース分類
上記の機械学習を使った研究では,いずれもテキストのみの情報を対象としている.
別の対象として,テキスト・画像を併せた情報を分類する機械学習モデルの検討も数多く行われている.
大まかな形としては,まずテキスト・画像を何らかの方法でベクトル化する.
その後2種のベクトルを結合し,真偽判定を行うモデルに渡す形をとっている.
例えばJinらの研究\cite{jin2017multimodal}では,テキストではLSTM,画像ではVGG-19を使用してベクトル化しており,
更にAttentionとソーシャルコンテキスト(ハッシュタグ,URL等)によって更に高精度な分類を行うモデルが提案されている.
またWangらの研究\cite{wang2018eann}では,EANNというモデルが提案されている.
これは画像のベクトル化においては同じくVGG-19を使用しているが,テキストではテキストCNNを使用している.
% このモデルの大きな特長として,敵対的生成ネットワーク(GAN)を模倣する形をとることが挙げられる.
% ニュースを学習する際には,どうしても扱うニュースが扱うイベントの偏りによる影響を受けてしまう.
% このイベントによる特殊性を排するために,真偽分類に加えて扱われたイベントも分類することによって,
% 特徴化する際にイベントによる特殊性を排するアプローチが行われている.
% 実際にこれによって分類精度が改善した点が上記研究によって報告されている.
% * GANは当研究では使わないので記述から外すことにした

\section{研究課題}
上記のEANNモデルのような画像・テキスト双方を扱うモデルでは,実際に真実・フェイクとのカテゴリ分類において画像単独・テキスト単独の分類に比べて優秀な成績を収めている\cite{wang2018eann}.\@
しかしながら,あくまで``真実なのかそうでないのか''という2カテゴリで分類しているため,
``他者を欺くための情報なのか,皮肉・風刺を込めた情報なのか''という観点での分析がなされていない.

% 本研究
本研究では,画像つきで発信された情報に対して,正しい情報か・フェイクニュースか・ジョークニュースかを判断するモデルを構築する.
このモデルを使い,従来から画像・テキスト複合のデータセットに対して3カテゴリでも優秀な分類が行えることを示すことを目指す.
それにより,SNSユーザの情報収集を支援するエージェントの開発につなげることが可能となる.

% 実験
上記の提案する情報分類システムを検証するために,
事前に用意されたデータセットを用いて10 分割交差検定によって分析を行う.
また上記システムの分類性能を評価するために,
画像・テキスト単独で分類を行った結果と比較することで,
提案システムが目標に適していることを示す.
その結果テキスト単独でのマクロF値が約0.22,画像単独でのマクロF値が約0.40であったのに比べ,
提案モデルのマクロF値は約0.93という数値を出し,提案モデルの有効性が示された.

% \bibliography{../ref/ref}
