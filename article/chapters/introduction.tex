% 背景[フェイクニュースの自動判別の必要性+ジョークニュースを区別する必要性]+想定環境の説明(導入との被りに注意)
\chapter{序論}
%
\section{背景}
% 背景

%想定環境

\section{先行研究}


\section{課題}
上記のEANNモデルのような画像・テキスト双方を扱うモデルでは,実際に真実・フェイクとのカテゴリ分類において画像単独・テキスト単独の分類に比べて優秀な成績を収めている\cite{eann}.
しかしながら,あくまで``真実なのかそうでないのか``という2カテゴリで分類しているため,
``他者を欺くための情報なのか,皮肉・風刺を込めた情報なのか``という観点での分析がなされていなかった.

%本研究
本研究では,画像つきで発信された情報に対して,正しい情報か・フェイクニュースか・ジョークニュースかを判断するモデルを構築した.
このモデルを使い,従来から画像・テキスト複合のデータセットに対して3カテゴリでも優秀な分類が行えることを示すことを目指した.
それにより,SNSユーザの情報収集を支援するエージェントの開発につなげることが可能となる.

%実験
上記の提案する情報分類システムを検証するために,
事前に用意されたデータセットを用いて10 分割交差検定によって分析を行う.
また上記システムの分類性能を評価するために,
画像・テキスト単独で分類を行った結果と比較することで,
提案システムが目標に適していることを示す.

% \bibliography{../ref/ref}
