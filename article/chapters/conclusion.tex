%
\chapter{おわりに}
%
\section{本論文のまとめ}
本研究では,SNS上で画像と文章を併せて発信された情報に対して,正しいニュース・フェイクニュース・ジョークニュースを判断するモデルを提案した.
実際に3カテゴリ分類を行った結果,文章・画像単体から分類した場合に比べて,全ての評価指標において非常に優秀な分類成績を挙げた.
これによりSNS上における画像つき投稿に対して,ジョークニュースを含めた3カテゴリ分類も有効であることが示された.
%
\section{今後の展望}
このモデルの発展形として,いくつかの方法が考えられる.

例えば文章特徴生成器に対して,テキストCNNではなくVosoughiらの研究\cite{Vosoughi:2016:TLT:2911451.2914762}によってSNS投稿を分析するために提案された,
文字単位でベクトル変換する方法を採用することなどが考えられる.

また,データセットが扱う出来事やイベントによる特殊性の対策として,
Wangらの研究\cite{wang2018eann}では敵対的生成ネットワーク(GAN)を模倣する形をとることが挙げられていた.
このイベントや出来事による特殊性を排するために,真偽分類に加えて扱われたイベントも分類することによって,
特徴化する際に特殊性を排し,フェイクニュースの普遍的な特徴を抽出するようなアプローチが行われていた.
実際にこれによって分類精度が改善した点が上記研究によって報告されていたため,当研究でも有効に働く可能性がある.

提案手法を日本語投稿に対応させることを考えた場合,
まずSNS上で日本語による画像つきの正しいニュース・フェイクニュース・ジョークニュースの投稿を収集する必要があると考えられる.
その上に形態素解析によって分かち書きする部分を加え,さらに日本語用の事前学習済みword2vecを用意する必要だと考えられる.
もしも既に日本語投稿による3カテゴリ分類済みのデータセットがあれば投稿を収集する必要はないが,
残念ながら国内に今回使用したデータセットに近い規模をもつものがないのが現状である.

% 