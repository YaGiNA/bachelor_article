%
\chapter{評価実験}
%
\section{データセット}
今回の実験での訓練データセットでは,
Boididouらの研究\cite{boididou2015verifying}によって提案されたTwitter投稿データセットを使用する.
こちらもTwitter上でフェイクニュースを検出するために作られたデータセットであるが,
付加されたラベルとしてReal, Fake, そしてHumorがあり,
ジョークニュースを含めた3カテゴリ分類に適したものとなっているため,当研究で採用する.
データセットでは訓練用と検証用として2部に分かれていたが,
当研究では訓練用とされた部分を対象に10分割交差検定することにする.
データセット内ではツイート文章と画像のみならず,
タイムスタンプや投稿者といったソーシャルコンテキスト情報も含まれている.
当研究ではソーシャルコンテキストは対象に含まず,文章と画像のみで3カテゴリ分類することを目指す.
% 
\section{比較対象手法}
% 
\section{実験条件}
% 
\section{実験結果}
% 
