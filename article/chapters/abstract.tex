\chapter{概要}

%背景
SNSの発展によりあらゆる情報入手が容易になった反面,人を欺くために故意に作成された虚偽の情報であるフェイクニュースが社会問題になっている.
特に画像と併せて発信されたものは,テキストのみならず画像と併せた分析アプローチが有効である.
虚偽の情報としては,もう1つジョークニュースというものもある.
これは人を欺くためではなく,社会風刺や皮肉のために作られた情報という特徴がある.
しばしばこの2カテゴリが混同され,ジョークニュースが批判に晒されることがあることも問題となっている.

%既存課題
既にテキスト・画像をニューラルネットワークの一種であるCNNによって分析して真偽を判定する自動判別モデルが提案されている.
実際に真実・フェイクとのカテゴリ分類において優秀な成績を収めているものの,
ジョークとしての嘘情報と人を欺くための嘘情報が区別されていない.


%提案
本研究では,正しい情報・ジョークニュース・フェイクニュースの3カテゴリを分類することで,より画像つきフェイクニュースの検出精度を向上させることを目指した.


%実験結果
実際にSNSから収集した画像つきのデータセットを対象にカテゴリ分類を行った結果,3カテゴリでもマクロF値が約0.93と良好な結果を示した.


%