\chapter{評価}

\section{考察}
今回の評価実験では,提案手法が3指標全てにおいて比較対象手法より優れた分類成績を収めた.
これにより,SNS上で画像つきの投稿を対象にした場合,正しいニュース・フェイクニュースの分類タスクのみならず,
ジョークニュースも含めた分類においても従来のマルチメディアモデルのアプローチが有効であることが示唆されたのではないかと考えられる.

また比較対象手法に限って結果を観察すると,文章単体より画像単体の分類の方が優秀な分類成績であった.
これは自然言語より画像の方が分類タスクにおいて研究が進んでいることや,
SNS上の投稿であった故に単語埋め込みに変換する際に\texttt{<unknown>}に変換されやすい傾向にあったことや,
文章の場合英語以外の投稿に対応できないものの,画像においては英語圏以外の投稿であっても十分言語の違いに影響されにくかったことなど,
いくつかの原因が推察される.

\section{課題}


%