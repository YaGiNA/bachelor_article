\chapter{評価}\label{ch:evaluate}

\section{考察}
今回の評価実験では,提案手法が3指標全てにおいて比較対象手法より優れた分類成績を収めた.
これにより,SNS上で画像つきの投稿を対象にした場合,正しいニュース・フェイクニュースの分類タスクのみならず,
ジョークニュースも含めた分類においても従来のマルチメディアモデルのアプローチが有効であることが示唆されたのではないかと考えられる.

また比較対象手法に限って結果を観察すると,文章単体より画像単体の分類の方が優秀な分類成績であった.
これは自然言語より画像の方が分類タスクにおいて研究が進んでいることや,
SNS上の投稿であった故に単語埋め込みに変換する際に\texttt{<unknown>}に変換されやすい傾向にあったことや,
文章の場合英語以外の投稿に対応できないものの,画像においては英語圏以外の投稿であっても十分言語の違いに影響されにくかったことなど,
いくつかの原因が推察される.

\section{課題}
今回分類するにあたり,大きな課題となったのが文章投稿の単語埋め込みへの変換であった.
例えば今回使用したデータセットがTwitterから収集されたものであったため,
事前学習済みword2vecモデルが対応できない短縮語や造語(ハッシュタグなど)といったユーザ生成コンテンツに対応することが難しかった.

また,このモデルに限らずフェイクニュース検出というタスクにおいては,Wangらの研究\cite{Wang:2018:EEA:3219819.3219903}によってある問題点が指摘されていた.
訓練に使ったデータセットが扱うイベントや出来事の特殊性の影響を受けることにより,
検証する時に訓練になかった別のイベントや出来事が使われた場合に正常な判断ができなくなる点であった.

さらに,このモデルは英語のみを対象としたものであった点も挙げられた.
データセット内一部では他国の言語が含まれていたため,単語埋め込みに変換する際に大幅に\texttt{<unknown>}に変換される傾向もあった.
%