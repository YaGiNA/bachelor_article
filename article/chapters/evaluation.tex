\chapter{評価}

\section{考察}
今回の評価実験では,提案手法が3指標全てにおいて比較対象手法より優れた分類成績を収めた.
これにより,SNS上で画像つきの投稿を対象にした場合,正しいニュース・フェイクニュースの分類タスクのみならず,
ジョークニュースも含めた分類においても従来のマルチメディアモデルのアプローチが有効であることが示唆されたのではないかと考えられる.

また比較対象手法に限って結果を観察すると,文章単体より画像単体の分類の方が優秀な分類成績であった.
これは自然言語より画像の方が分類タスクにおいて研究が進んでいることや,
SNS上の投稿であった故に単語埋め込みに変換する際に\texttt{<unknown>}に変換されやすい傾向にあったことや,
文章の場合英語以外の投稿に対応できないものの,画像においては英語圏以外の投稿であっても十分言語の違いに影響されにくかったことなど,
いくつかの原因が推察される.

\section{課題}
今回分類するにあたり,大きな課題となったのが文章投稿の単語埋め込みへの変換であった.
例えば今回使用したデータセットがTwitterから収集されたものであったため,
事前学習済みword2vecモデルが対応できない短縮語や造語(ハッシュタグなど)といったユーザ生成コンテンツに対応することが難しかった.
これに対する策としては,例えばVosoughiらの研究\cite{Vosoughi:2016:TLT:2911451.2914762}によって提案された文字単位でベクトル変換する方法を採用することなどが考えられる.

また,このモデルに限らずフェイクニュース検出というタスクにおいては,Wangらの研究\cite{wang2018eann}によってある問題点が指摘されていた.
訓練に使ったデータセットが扱うイベントや出来事の特殊性の影響を受けることにより,
検証する時に訓練になかった別のイベントや出来事が使われた場合に正常な判断ができなくなる点であった.
上記研究ではこの対策として敵対的生成ネットワーク(GAN)を模倣する形をとることが挙げられていた.
このイベントや出来事による特殊性を排するために,真偽分類に加えて扱われたイベントも分類することによって,
特徴化する際に特殊性を排し,フェイクニュースの普遍的な特徴を抽出するようなアプローチが行われていた.
実際にこれによって分類精度が改善した点が上記研究によって報告されているため,当研究でも有効に働く可能性がある.

さらに,このモデルは英語を対象としたものであった.
もし提案手法を日本語投稿に対応させるためには,まず日本語による画像つき正しいニュース・フェイクニュース・ジョークニュースの投稿を収集し,
形態素解析によって分かち書きする部分を加え,さらに日本語用の事前学習済みword2vecを用意する必要がある.
もしも既に日本語投稿による3カテゴリ分類済みのデータセットがあれば投稿を収集する必要はないが,
残念ながら国内に今回使用したデータセットに近い規模をもつものがないのが現状である.

%