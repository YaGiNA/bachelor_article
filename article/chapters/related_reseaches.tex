\chapter{関連研究}
\section{webページの体裁による分類}
福島らの研究\cite{2007}では,webページの体裁から信憑性を評価するモデルが提案された.
これは,情報そのものではなく情報が掲載されているwebページがどういった形式やコンテンツを持っているかをアンケート調査によって判断するものであった.
例えば,管理者の連絡元を表記したり,記事の公開・更新日時が明記されていたりしていた場合は,掲載情報の信憑性にポジティブな影響を与えていることが確認された.
逆に,掲載リンクが機能していなかったり,広告が1つ以上表示されていたりしていた場合は,掲載情報の信憑性にネガティブな影響を与えることが確認された.

\section{画像・文章の分析}
画像分類のは近年目まぐるしい発展を遂げた.
特に画像の被写体から分類するタスクにおいては,
VGG19のように16-19層の畳み込み層を取り入れたモデルが非常に高い分類成果を挙げることが報告\cite{DBLP:journals/corr/SimonyanZ14a}された.
また,VGG19を含めた多くのモデルでは,事前学習済みモデルが配布されているため,自分で転移学習を行うことも容易である.

フェイクニュースに限らず,文章を対象とした分類はいくつかのタスクがある.
例えば,文章から執筆者の感情を判断するセンチメント分析や,
ニュース記事から該当するカテゴリを判断するカテゴリ分類などがある.
当研究では,分類先のカテゴリが3種類に固定されているため,カテゴリ分類の一環とみなすことができる.
機械学習によって分類する場合,第\ref{ch:introduction}章の通り非常に数多くの手法が使われてきた.
最近では,ニューラルネットワークを活用して人間の短期記憶を再現したLSTM\cite{7508408}では,
人間の短期記憶を再現することによって,分類のみならず文章を生成するタスクにおいても発展している.
また,上記ではGPUによる並列実行が難しいため,
画像と同じく並列実行が可能なCNNをテキスト用にアレンジしたテキストCNNも提案\cite{DBLP:journals/corr/Kim14f}され,
広く使われている.

画像と文章を組み合わせた研究も数多くなされてきた.
例えば,画像をCNNで分析してLSTMによってキャプションを生成する研究\cite{7298935}によって,
より精度の高いキャプション生成ができたことが報告された.
キャプション生成のほかに,
画像に対して文章で視覚質問(画像に写ったものを問う)に応答することを目的としたVQA\cite{7410636}というモデルも提案された.

\section{フェイクニュース対策}
現在,フェイクニュースを判断する手法の1つに有識者によって事実関係を確認するファクトチェックがある.
例えばPolitifact.comではTruth-o-meterという独自指標によって,
政治的主張に対して疑わしさを7段階で評価\cite{holan_2018}している.
その中では,真実ではあるが重要な部分を省くことによて誤解を招きやすい``half-true''や,
一部真実を含むものの,重要な事実が無視されていることを示す``mostly-false'',
主張が正確ではない``false'',
そして完全に虚偽であり,ばかげた主張とする``pants-on-fire''など,
多くの評価名が用意されている.

フェイクニュース自体への対策が発展していく中で,
フェイクニュースを``真実か嘘か''という基準から判断すること自体に疑義を唱える取り組みも存在する.
現在,Mike Tamir氏によって立ち上げられたFakerFactというwebアプリケーションがある\cite{tamir}.
この取組では,フェイクニュースはセンセーショナルな文章を書くことによって読者の本能に働きかけ,
読者に拡散させる扇動を目的としていることに着目していた.
このwebサイトではWaltという独自のAIを搭載しており,
文章を``真実か嘘か''は判断せず,文章の論調から以下の6カテゴリに分類していた.

\begin{itemize}
    \item Journalism: 事実をベースとした文章
    \item Wiki: 辞書的文章
    \item Agenda Driven: 何かしらの意図がみられる文章
    \item Opinion: 意見が書かれた文章
    \item Sensational: 扇動を目的とした文章
    \item Satire: 風刺・皮肉
\end{itemize}

あくまで真偽は判断せず読者に何を伝えたいのかを類推することで,
読者が真偽を判断する手助けになることがこのモデルの目的であった.
このモデルでも,Satireとして風刺・皮肉をもつ文章(ジョークニュース)が区別されてあった.

このようにフェイクニュースを判断するにあたって,
近年では``真実か嘘か''という観点にとらわれない多くのアプローチや分類が行われていることがわかる.
