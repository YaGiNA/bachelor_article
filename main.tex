\documentclass[a4paper,12pt,oneside,openany]{jsbook}

\usepackage[papersize={210truemm, 297truemm}]{geometry}
\geometry{top=40truemm,bottom=35truemm,left=25truemm,right=25truemm}
%
\usepackage[utf8]{inputenc}
\usepackage{booktabs}
\usepackage{url}
\usepackage{nidanfloat}
\usepackage{afterpage}
\usepackage{setspace}
\usepackage{multirow}
\usepackage{here}

\usepackage{amsmath,amssymb}
\usepackage{bm}
%\usepackage{graphicx}
\usepackage[dvipdfmx]{graphicx}
\usepackage{subfigure}
\usepackage{verbatim}
\usepackage{wrapfig}
\usepackage{ascmac}
\usepackage{makeidx}

\usepackage{algorithm}
\usepackage{algorithmic}
%
\setcounter{tocdepth}{3}
%
\makeindex
%余白設定
\setlength{\textwidth}{\fullwidth}
\setlength{\textheight}{35\baselineskip}
\addtolength{\textheight}{\topskip}
\setlength{\voffset}{-0.55in}
\setlength{\abovedisplayskip}{2.5pt} % 上部のマージン
\setlength{\belowdisplayskip}{2pt} % 下部のマージン
%
\newcommand{\etal}{\textit{et al}.,}
\newcommand{\ie}{\textit{i}.\textit{e}.,}
\newcommand{\eg}{\textit{e}.\textit{g}.,}
\newcommand{\reffig}[1]{図\ref{#1}}
\newcommand{\reftab}[1]{表\ref{#1}}
\newcommand{\refequ}[1]{式\ref{#1}}
\newcommand{\refsec}[1]{\ref{#1}節}

%%%%%%%%%%%%%%%%%%%%%%%%%%%%%%%%%%%%%%%%%%%%%%%%%%%%%%
%\title{タイトル}
%\author{氏名}
%\date{\today}
\begin{document}
%
%\maketitle
%
\frontmatter
%表紙
\begin{titlepage}
  \vspace*{7mm}
  \begin{center}
    \Large{平成31年度卒業論文}\\

    \vspace{40truemm}

    \LARGE{\textbf{画像付きフェイクニュースとジョークニュースの検出・分類に向けた機械学習モデルの検討}}\\

    \vspace{30truemm}

    \Large{電気通信大学 情報理工学部 総合情報学科}\\
    \Large{メディア情報学コース}\\

    \vspace{15truemm}

    \begin{table*}[h]
        \centering
        \hspace*{4em}
        \begin{tabular}{rcl}
            \large{学籍番号}&\large{:}&\large{1510151} \\
            \large{氏名}&\large{:}&\large{栁 裕太} \vspace{5truemm} \\
            \large{主任指導教員}&\large{:}&\large{田原 康之 准教授} \vspace{5truemm} \\
            \large{指導教員}&\large{:}&\large{大須賀 昭彦 教授} \vspace{5truemm} \\
            \large{指導教員}&\large{:}&\large{清 雄一 准教授} \vspace{5truemm} \\
            \large{提出年月日}&\large{:}&\large{平成31年2月8日(金)} \\
        \end{tabular}
    \end{table*}
  \end{center}
\end{titlepage}
%
%アブスト
\addcontentsline{toc}{chapter}{概要}

\chapter{概要}

%背景


%既存課題


%提案


%実験結果



%
\tableofcontents
%
\mainmatter
%本文
% 背景[フェイクニュースの自動判別の必要性+ジョークニュースを区別する必要性]+想定環境の説明(導入との被りに注意)
\chapter{序論}
%
\section{背景}
% 背景

%想定環境

\section{先行研究}
%フェイクニュースのテキストによる分類
フェイクニュースに限らず,風評やwebページの信憑性を評価するモデルの構築の研究は数多く行われている.
例えば,福島らの研究\cite{fuk}では,webページの体裁から信頼性を評価するモデルが提案されている.
またWuらの研究\cite{wu}では,SNS上で拡散された情報に対して,``誰が・どのような経緯で拡散したか''という情報から信憑性を判断するモデルも提案されている. 
また教師あり・なし問わず,機械学習による分類が非常に盛んに行われている.
なかでもGranikらの研究\cite{gra}やGildaの研究\cite{gil},そして松尾の研究\cite{mat}により,
ナイーブベイズ分類器やSVM,決定木といった教師あり学習によるフェイクニュースや流言を分類するタスクにおいて,
優秀な分類成果を挙げることが報告されている.

%マルチメディアでフェイクニュース分類

\section{課題}
上記のEANNモデルのような画像・テキスト双方を扱うモデルでは,実際に真実・フェイクとのカテゴリ分類において画像単独・テキスト単独の分類に比べて優秀な成績を収めている\cite{eann}.
しかしながら,あくまで``真実なのかそうでないのか``という2カテゴリで分類しているため,
``他者を欺くための情報なのか,皮肉・風刺を込めた情報なのか``という観点での分析がなされていない.

%本研究
本研究では,画像つきで発信された情報に対して,正しい情報か・フェイクニュースか・ジョークニュースかを判断するモデルを構築する.
このモデルを使い,従来から画像・テキスト複合のデータセットに対して3カテゴリでも優秀な分類が行えることを示すことを目指す.
それにより,SNSユーザの情報収集を支援するエージェントの開発につなげることが可能となる.

%実験
上記の提案する情報分類システムを検証するために,
事前に用意されたデータセットを用いて10 分割交差検定によって分析を行う.
また上記システムの分類性能を評価するために,
画像・テキスト単独で分類を行った結果と比較することで,
提案システムが目標に適していることを示す.

% \bibliography{../ref/ref}

%
\chapter{提案手法}
%
\section{エージェント詳細}
% 
本章では韻律的特徴量から発話者の感情を推定するシステム,及び,それを使用した
応答変更機能のある対話エージェントの提案を行う.提案手法のエージェントの応答の流れを図\ref{fig:system_pic31}に示す.

\begin{figure}[h]
	\centering
		\fbox{\includegraphics[width=\fullwidth]{./img/agent.png}}
        \caption{提案手法のエージェントの応答の流れ}
        \label{fig:system_pic31}
	\centering
\end{figure}

ユーザは初めにエージェントに向かって音声を入力する.その後エージェントは入力された音声から韻律的特徴量を抽出し,識別機にその特徴量を入力する.識別機はその特徴量から話者の感情を推定し,その推定底結果からエージェントは応答を生成し,音声として出力する.
%
%
\section{韻律情報}
韻律情報とは音声の高さや大きさ,速さ,長さ,イントネーション等,文字に現れない話者の発話の特徴である.これらの情報は個人により異なり,また一人ひとりの発話の方法でも差異が生じてくる.この差異を数としてとらえるために使用するものが韻律的特徴量であり,後述する感情推定システムで使用される.

\section{韻律的特徴量を用いた感情推定システム}
%
\subsection{韻律的特徴量}
韻律的特徴量とはパワーやピッチといった音声に関する特徴量のことを指す.
本システムで使用する韻律的特徴量はINTERSPEECH 2009 Emotion Challenge\cite{tokutyou}で使用された特徴量を用いた.表\ref{tokutyou}に一覧を示す.

\begin{table}[h]
\begin{center}
%韻律的特徴量
	\caption{韻律的特徴量}
	\begin{tabular}{c|c|c} \hline
		特徴量名 & $\Delta$成分 & 統計量(12種) \\ \hline \hline
		RMS energy & $\Delta$ RMS energy & amean \\ \cline{1-2}%1
		F0 & $\Delta$ F0 \hline & standard deviation \\ \cline{1-2}%2
		ZCR & $\Delta$ ZCR \hline & max, maxPos, min, minPos \\ \cline{1-2}
		voice Prob & $\Delta$ voice Prob \hline  & range, skewness, kurtosis \\ \cline{1-2}
		MFCC 1-12 & $\Delta$ MFCC 1-12 \hline & linregc 1, linregc 2, linregerrQ \\ \hline%3
	\end{tabular}
	\label{tokutyou}
\end{center}
\end{table}

特徴量の具体的な内容は,各フレームごとの平均パワー,ピッチ周波数,ゼロ交差率,Voice probability(全パワーに占める調波成分の割合),
12次元のMFCCの計16次元とその{\Delta}成分が16次元,
さらにその32次元に対し12種類の統計量(平均,標準誤差,最大値とその位置,最小値とその位置,値のレンジ,尖度,歪度,
1次の回帰係数とその切片,回帰誤差)をとった32*12=384次元である.
%+特徴量説明

韻律的特徴量の抽出にはオープンソフトウェアのであるOpenSMILE\cite{opensmile}を使用した.OpenSMILEはミュンヘン工科大学が開発した,「音声認識」,「音楽認識」,「パラ言語認識」等の研究向けに作られた音声特徴量抽出ソフトウェアである.

%
%\newpage
%
\subsection{識別機の作成}
識別機は入力を発話の韻律的特徴量,出力を発話者の感情値とする.
識別機の作成にはオープンソースの機械学習ソフトウェアであるWeka\cite{weka}のSupport Vector Regression(SVR)を用いた.
%+SVRの説明
また学習用の対話コーパスは宇都宮大学パラ言語情報研究向け音声対話データベース\cite{taiwa}を使用した.
このデータベースは親しい間柄の6ペアの発話内容が記録されており,全部で4840発話が収録されている.
発話には快-不快,覚醒-睡眠,支配-服従,信頼‐不信,関心‐無関心,肯定的,否定的の6感情についてラベルが付与されている.

識別機の作成は,学習用対話コーパスの発話群の韻律的特徴量を抽出し,それを付与されている6感情それぞれについてSVRによって学習させることで作成した.

\section{感情に応じた応答生成}
識別機によって出力された感情値を用いて応答の変更を行う.
ある感情の推測値に対して閾値を設ける.それぞれの感情の推測値がその閾値を上回るかどうかを判定し,上回った感情の組み合わせによって応答を変更し生成する.
これにより,文字等の言語情報を用いずに応答を変更することができる.

%図の挿入


%
%
\newpage
%
%
%
%
%
%
%
%
%
%
% 
%
\chapter{評価実験}
%
\section{データセット}
今回の実験での訓練データセットでは,
Boididouらの研究\cite{boididou2015verifying}によって提案されたTwitter投稿データセットを使用した.
こちらもTwitter上でフェイクニュースを検出するために作られたデータセットであるが,
付加されたラベルとしてReal, Fake, そしてHumorがあり,
ジョークニュースを含めた3カテゴリ分類に適したものとなっているため,当研究で採用した.
データセットでは訓練用と検証用として2部に分かれていたが,
当研究では訓練用とされた部分を対象に10分割交差検定することにした.
データセット内ではツイート文章と画像のみならず,
タイムスタンプや投稿者といったソーシャルコンテキスト情報も含まれている.
当研究ではソーシャルコンテキストは対象に含まず,文章と画像のみで3カテゴリ分類することを目指した.
% 
\section{比較対象手法}
今回,画像つき文章投稿を3カテゴリに分類する提案手法の有効性を調べるために2種類の比較対象手法を用意した.
1つは文章のみで投稿を分類する手法(以降,Textと表記),もう1つは画像のみで投稿を分類する手法(以降,Imageと表記)であった.
いずれも上記提案モデルから文章・画像特徴生成器を除外したモデルを使用した.
またTextは入力データを提案モデルが使用したデータセットから画像を削除したものを使用した.
Imageは全投稿で使用された画像を対象とし,同じ画像に対して複数の文章投稿があった場合は1件として数えることにした.
% 
\section{実験条件}
\subsection{使用データ統計}
上記の条件を踏まえ,提案手法・Text・Imageが扱う3カテゴリの投稿件数は以下の表\ref{table:posts}の通りである.
Textが使用するデータは提案手法が扱うデータから画像を削除したものであるため,提案手法と全く同じ件数になった.
Imageは,同じ画像に対して複数の投稿があったため,他2手法と比べて非常に少なくなっている.

\begin{table}[h]
    \caption{提案手法と比較対象手法が扱うカテゴリ毎の投稿数}
    \label{table:posts}
    \centering
    \begin{tabular}{clll}
        \hline
        手法 & Real & Fake & Humor \\
        \hline \hline
        Text & 3021 & 4233 & 1509 \\
        Image & 172 & 157 & 82 \\
        提案手法 & 3021 & 4233 & 1509 \\
        \hline
    \end{tabular}
\end{table}

%
\subsection{モデル条件}
\subsubsection{Text}
まず単語埋め込みに変換する際,Google Newsデータセットから事前学習済みのword2vecモデル\cite{google_2013}を使用した.
このモデルでは,各単語を300次元のベクトルに変換するものであった.
ここでwod2vecモデルに該当しない単語が出現した場合,\texttt{<unknown>}としてseed値固定ランダムベクトルを生成することにした.
また,投稿の全単語中50\%以上が\texttt{<unknown>}の場合,実態に則さない学習を避けるために学習対象から外すことにした.
その後テキストCNNに送られ,1つの文章に対して1つの300次元のベクトルが生成され,ニュース分類器に渡す形となった.
なお,フィルタサイズは2-5までとし,隠れ層は1つ用意し,ユニット数は300とした.
隠れ層では60\%のユニットが無視されるDropoutを導入した.
ニュース分類器内でも隠れ層は1つ用意し,ユニット数は300,上記と同じ条件のDropoutも導入した.
%
\subsubsection{Image}
モデルの複雑化を避けるため,1つの投稿に複数枚画像が付加されていた場合は最初の1枚のみをモデルに入力させることにした.
画像は事前学習済みVGG-19モデルに入力し,1つの画像に対して1つの300次元のベクトルが生成され,ニュース分類器に渡す形となった.
本来のVGG-19は最終層にて1000次元のベクトルが出力されるが,最終層のみ改変して300次元のベクトルが出力されるようにした.
また前記の通り過学習を避けるために最終層を除き事前学習済みの状態を維持させることにした.
ニュース分類器の条件は上記Textと同一であった.
%
\subsubsection{提案手法}
提案手法では,TextとImageを統合した形をとったため,画像・文章の部分は上記と同様の条件をとった.
画像・文章の特徴を結合するため,ニュース分類器に渡されるのは600次元のベクトルであった.
それにあわせ,隠れ層のユニット数も600とした.
%
\subsection{評価指標}
評価指標では,Precision(精度), Recall(再現率), F値(左2値の調和平均)を使用することにした.
算出する方法上各カテゴリ毎に上記指標があるが,今回使用するデータセットでは極端にカテゴリが偏っていないので,
各カテゴリの指標を先に算出してから3カテゴリの平均をとるマクロ平均を評価に使うことにした.

\section{実験結果}
3モデルに対して10分割交差検定を行った結果が以下の表\ref{table:result}の通りである.
% 
\begin{table}[h]
    \caption{各モデルの分類成果(マクロ平均)}
    \label{table:result}
    \centering
    \begin{tabular}{clll}
        \hline
        手法 & Precision & Recall & F値 \\
        \hline \hline
        Text & 0.3649 & 0.3677 & 0.3016 \\
        Image & 0.4942 & 0.5055 & 0.4667 \\
        提案手法 & 0.9268 & 0.9362 & 0.9286 \\
        \hline
    \end{tabular}
\end{table}

この結果を見ると,提案手法が他2手法と比べて非常に高い分類成果を挙げたことが読み取れた.
また,比較対象手法内で比べると画像単体の方が分類成果が良好である点もみられた.

\chapter{評価}\label{ch:evaluate}

\section{考察}
今回の評価実験では,提案手法が3指標全てにおいて比較対象手法より優れた分類成績を収めた.
これにより,SNS上で画像つきの投稿を対象にした場合,正しいニュース・フェイクニュースの分類タスクのみならず,
ジョークニュースも含めた分類においても従来のマルチメディア手法のアプローチが有効であることが示唆されたのではないかと考えられる.

また比較対象手法に限って結果を観察すると,文章単体より画像単体の分類の方が優秀な分類成績であった.
これは自然言語より画像の方が分類タスクにおいて研究が進んでいることや,
SNS上の投稿であった故に単語埋め込みに変換する際に\texttt{<unknown>}に変換されやすい傾向にあったことや,
文章の場合英語以外の投稿に対応できないものの,画像においては英語圏以外の投稿であっても十分言語の違いに影響されにくかったことなど,
いくつかの原因が推察される.

提案手法AとBを比較すると,3指標において明確な差はみられなかった.
この点より画像つき投稿を3カテゴリ分類するタスクにおいて,
細かい手法問わず文章・画像それぞれを特徴化・結合して分類することが有効であることが示された.

また表\ref{table:detail}の各カテゴリ毎の成績を見ると,更に興味深い点がいくつか見えてくる.

例えばJokeの分類成績が全手法において一貫してReal\&Fakeより劣悪なものとなった.
これは3カテゴリ内でJokeが最も正しく分類することが困難であることを示唆している.

Textの結果に目を向ける.先述の通りJokeのRecallが0.03と極端に低い結果となり,Real\&FakeのPrecisionが同じく0.52であった.
このことからTextでは正解がJokeである投稿の大多数を見抜くことができず,一部は誤ってFakeと扱われていた可能性が示された.

\section{課題}
今回分類するにあたり,大きな課題となったのが文章投稿の単語埋め込みへの変換であった.
例えば今回使用したデータセットがTwitterから収集されたものであったため,
事前学習済みword2vec手法が対応できない短縮語や造語(ハッシュタグなど)といったユーザ生成コンテンツに対応することが難しかった.

また,この手法に限らずフェイクニュース検出というタスクにおいては,Wangらの研究\cite{Wang:2018:EEA:3219819.3219903}によってある問題点が指摘されていた.
訓練に使ったデータセットが扱うイベントや出来事の特殊性の影響を受けることにより,
検証する時に訓練になかった別のイベントや出来事が使われた場合に正常な判断ができなくなる点であった.

さらに,この手法は英語のみを対象としたものであった点も挙げられた.
データセット内一部では他国の言語が含まれていたため,単語埋め込みに変換する際に大幅に\texttt{<unknown>}に変換される傾向もあった.
%
%
\chapter{おわりに}
%
\section{本論文のまとめ}
%
\section{今後の展望}
% 

%謝辞
\chapter*{謝辞}
\addcontentsline{toc}{chapter}{謝辞}
本研究を行うにあたり,ご多忙の中,終始適切かつ丁寧なご指導をして下さった大須賀昭彦教授,田原康之准教授,清雄一准教授に深く感謝致します.貴重な勉学の機会を与えてくださったことに深く御礼申し上げます.
また,研究の機会と議論・研鑽の場を提供して頂き,活発な議論と貴重なご意見を頂いた研究グループの皆様,大須賀・田原・清研究室の皆様に感謝の意を表します.

%\chapter*{研究業績}
%\addcontentsline{toc}{chapter}{研究業績}

%\section*{国際会議}
%\begin{achievement}
%\item \underline{\textbf{Minato Sato}}, Ryohei Orihara, Sei Yuichi, Yasuyuki Tahara and Akihiko Ohsuga: Japanese Text Classification by Character-Level Deep ConvNets and Transfer Learning, The 9th International Coneference on Agents and Artificial Intelligence (ICAART2017), Feb 2017. (accepted as a Full Paper)
%\end{achievement}

%\section*{査読付き国内シンポジウム・ワークショップ}
%\begin{achievement}
%\item \underline{\textbf{佐藤挙斗}},折原良平,清雄一,田原康之,大須賀昭彦: 文字レベル深層学習による日本語テキストの分類と転移学習,合同エージェントワークショップ&シンポジウム2016 (JAWS2016),pp.199-206,2016年9月. (ショート発表) \textcolor{red}{{\bf 優秀発表賞}}
%\end{achievement}

%\section*{研究会}
%\begin{achievement}
%\item \underline{\textbf{佐藤挙斗}},折原良平,清雄一,田原康之,大須賀昭彦: 文字レベル深層学習の日本語データセットへの応用,第184回 情報処理学会 知能システム研究会 (SIG-ICS) ,2016年8月.
%\item \underline{\textbf{佐藤挙斗}},折原良平,清雄一,田原康之,大須賀昭彦: 文字レベル深層学習によるテキスト分類と転移学習,人工知能学会 第102回人工知能基本問題研究会(SIG-FPAI),2016年12月. 
%\end{achievement}

%参考文献
\newpage
% \bibliographystyle{plain}
% \bibliography{./tex/ref/ref}
%\bibliography{reference_slim}
%\bibliographystyle{plain}

\begin{thebibliography} {*}
%\bibitem{sibata} 柴田雅博,冨浦洋一,西口友美, 
%\textit{雑談自由対話を実現するためのWWW 上の文書からの妥当な候補文選択手法}
%	,人工知能学会論文誌,vol.24, no.6, pp.507-519, 2009.

%\bibitem{sugeo} LeFevre, K., DeWitt, D. and Ramakrishnan, R., 
%\textit{Mondrian Multidimensional K - Anonymity}
%	, Proc. IEEE ICDE, pp. 25 - 25, 2006.

\end{thebibliography}

%付録
%%
\appendix
\renewcommand{\thesection}{付録.\ \arabic{section}}
\renewcommand{\thesubsection}{\ \arabic{section}-\arabic{subsection}}
%
\section{統計表3  推計患者数,年齢階級・傷病大分類別}
\subsection{分類表1}
%
\begin{figure}[h]
  \begin{center}
    \includegraphics[width=\fullwidth]{./img/図1.jpg}
    %\caption{傷病大分類1}
  \end{center}
\end{figure}
%
\newpage
\subsection{分類表2}
%
\begin{figure}[h]
  \begin{center}
    \includegraphics[width=\fullwidth]{./img/図2.jpg}
    %\caption{傷病大分類2}
  \end{center}
\end{figure}
%
\newpage
\section{統計表9  総患者数,性・主な傷病別 }
%
\begin{figure}[h]
  \begin{center}
    \includegraphics[width=0.8\fullwidth]{./img/図3.jpg}
    %\caption{傷病大分類3}
  \end{center}
\end{figure}


%

\newpage
\printindex
%
%
\end{document}