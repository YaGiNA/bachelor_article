%
\chapter{提案手法}
%
\section{エージェント詳細}
% 
本章では韻律的特徴量から発話者の感情を推定するシステム,及び,それを使用した
応答変更機能のある対話エージェントの提案を行う.提案手法のエージェントの応答の流れを図\ref{fig:system_pic31}に示す.

\begin{figure}[h]
	\centering
		\fbox{\includegraphics[width=\fullwidth]{./img/agent.png}}
        \caption{提案手法のエージェントの応答の流れ}
        \label{fig:system_pic31}
	\centering
\end{figure}

ユーザは初めにエージェントに向かって音声を入力する.その後エージェントは入力された音声から韻律的特徴量を抽出し,識別機にその特徴量を入力する.識別機はその特徴量から話者の感情を推定し,その推定底結果からエージェントは応答を生成し,音声として出力する.
%
%
\section{韻律情報}
韻律情報とは音声の高さや大きさ,速さ,長さ,イントネーション等,文字に現れない話者の発話の特徴である.これらの情報は個人により異なり,また一人ひとりの発話の方法でも差異が生じてくる.この差異を数としてとらえるために使用するものが韻律的特徴量であり,後述する感情推定システムで使用される.

\section{韻律的特徴量を用いた感情推定システム}
%
\subsection{韻律的特徴量}
韻律的特徴量とはパワーやピッチといった音声に関する特徴量のことを指す.
本システムで使用する韻律的特徴量はINTERSPEECH 2009 Emotion Challenge\cite{tokutyou}で使用された特徴量を用いた.表\ref{tokutyou}に一覧を示す.

\begin{table}[h]
\begin{center}
%韻律的特徴量
	\caption{韻律的特徴量}
	\begin{tabular}{c|c|c} \hline
		特徴量名 & $\Delta$成分 & 統計量(12種) \\ \hline \hline
		RMS energy & $\Delta$ RMS energy & amean \\ \cline{1-2}%1
		F0 & $\Delta$ F0 \hline & standard deviation \\ \cline{1-2}%2
		ZCR & $\Delta$ ZCR \hline & max, maxPos, min, minPos \\ \cline{1-2}
		voice Prob & $\Delta$ voice Prob \hline  & range, skewness, kurtosis \\ \cline{1-2}
		MFCC 1-12 & $\Delta$ MFCC 1-12 \hline & linregc 1, linregc 2, linregerrQ \\ \hline%3
	\end{tabular}
	\label{tokutyou}
\end{center}
\end{table}

特徴量の具体的な内容は,各フレームごとの平均パワー,ピッチ周波数,ゼロ交差率,Voice probability(全パワーに占める調波成分の割合),
12次元のMFCCの計16次元とその{\Delta}成分が16次元,
さらにその32次元に対し12種類の統計量(平均,標準誤差,最大値とその位置,最小値とその位置,値のレンジ,尖度,歪度,
1次の回帰係数とその切片,回帰誤差)をとった32*12=384次元である.
%+特徴量説明

韻律的特徴量の抽出にはオープンソフトウェアのであるOpenSMILE\cite{opensmile}を使用した.OpenSMILEはミュンヘン工科大学が開発した,「音声認識」,「音楽認識」,「パラ言語認識」等の研究向けに作られた音声特徴量抽出ソフトウェアである.

%
%\newpage
%
\subsection{識別機の作成}
識別機は入力を発話の韻律的特徴量,出力を発話者の感情値とする.
識別機の作成にはオープンソースの機械学習ソフトウェアであるWeka\cite{weka}のSupport Vector Regression(SVR)を用いた.
%+SVRの説明
また学習用の対話コーパスは宇都宮大学パラ言語情報研究向け音声対話データベース\cite{taiwa}を使用した.
このデータベースは親しい間柄の6ペアの発話内容が記録されており,全部で4840発話が収録されている.
発話には快-不快,覚醒-睡眠,支配-服従,信頼‐不信,関心‐無関心,肯定的,否定的の6感情についてラベルが付与されている.

識別機の作成は,学習用対話コーパスの発話群の韻律的特徴量を抽出し,それを付与されている6感情それぞれについてSVRによって学習させることで作成した.

\section{感情に応じた応答生成}
識別機によって出力された感情値を用いて応答の変更を行う.
ある感情の推測値に対して閾値を設ける.それぞれの感情の推測値がその閾値を上回るかどうかを判定し,上回った感情の組み合わせによって応答を変更し生成する.
これにより,文字等の言語情報を用いずに応答を変更することができる.

%図の挿入


%
%
\newpage
%
%
%
%
%
%
%
%
%
%
% 